\newpage
\section{Introduction}
\label{sec:intro}

12.5\%

Describe the motivation for making red LEDs [500 Words max + Figures]

LEDs are more energy efficient than traditional incandescent lighting

Narrow wavelengths.

LEDs can be very small for their optical power.

LEDs can be switched on and off very quickly, this is useful for PWM control for dimmable bulbs.

Televisions

LEDs are cost effective for use on a large scale

\hl{The basic function of a Light-Emitting Diode is that two semiconducting layers of types P and N }

Longer Lifetime than incandescent

%
% \subsection{subsection1}
\label{sec:intro:subsection1}
\subsubsection{subsubsection1}
\label{sec:intro:subsubsection1}

Hello World!


%
% This is an example of a citation: \cite{lin_koizumi_yater_koeck_2014, emission_suppression_08}
%
%
%
% \begin{table}[ht]
% \centering
%  \begin{tabular}{| c | c | c | c |}
%  \hline
%  Semiconductor & Electron Mobility ($cm^2.V^{-1}.s^{-1}$) & Hole Mobility ($cm^2.V^{-1}.s^{-1}$) & Band Gap (eV) \\ %[0.5ex]
%  \hline\hline
%  Diamond & 4500 & 3800 & 5.47\\
%  Silicon & 1450 & 480 & 1.12\\
%  GaAs & 8500 & 400 & 1.43\\
%  GaN & 1100 & 200 & 3.45\\
%  \hline
% \end{tabular}
% \caption{Example table}
% \label{tab:example}
% \end{table}
%
%
%
%
% \begin{figure}[!htb]
%   \centering
%   \includegraphics[width=0.67\textwidth]{Figures/example.pdf}
%   \caption{Example Figure}
%   \label{fig:example}
% \end{figure}
%
% Here's an example equation:
%
% \begin{center}
% \begin{equation}
% \%_{err}(V) = \left(\frac{R(V)}{R_{avg}}-1\right)\times100
% \end{equation}
% \end{center}
%
% Here's an example plot with subplots:
%
% \input{Chapters/results_plotting/example/err.tex}
