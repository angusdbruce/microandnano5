\subsection{Cleanrooms}
\label{sec:fab:cleanrooms}

% \hl{3.125\%
%
% Briefly describe why fabrication takes place within a cleanroom. [100 words max]
%
% The clean rooms minimise contamination
%
% there are different scales of cleanroom and so the results from fabrication can be more predictable, e.g. in terms of the likelihood of devices having errors. <-- is this true?
%
% A cleanroom is a controlled environment }

Fabrication of electronic devices takes place in a cleanroom. Cleanrooms are controlled environments where atmospheric conditions are controlled. The class of a cleanroom specifies the number of particles of various sizes for a give volume of air. Guaranteeing the number and size of contaminating particles in a fabrication process gives a reasonable approximation as to the feasibility and the probability of success. As devices become smaller, the need for better cleanrooms increases since smaller particles will have a greater effect on the performance and yield of the devices.
