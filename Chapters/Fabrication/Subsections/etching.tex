\subsection{Etching}
\label{sec:fab:etching}

\hl{9.375\% In step 11, describe the choice of this etch process. What other process choices are there? Why do you think that this one was chosen? What advantages does this process have over other options? [250 words max]}

Before etching, the depth of the resist was found to be $2022nm$ by the stylus profiler. The exposed areas of the substrate were wet etched for 60 seconds using a H$_{2}$SO$_{4}$:H$_{2}$O$_{2}$:H$_{2}$O 1:8:40 solution. The resist depth was measured again and found to be $2273nm$ giving an etch depth of $251nm$. Since the thickness of the p+ GaAs contact layer is $150nm$, the etch will have gone $101nm$ into the p-InGaAsP layer assuming the layer thicknesses are exact.

Another etching process is dry etching where the etchant is either a gas or plasma. Dry etching can have a slower etch rate and poor selectivity or etch ratio.
